\section{if-conversion的发展简史}

if-conversion最早由Rice大学的Allen等人在1983年提出\cite{allen1983Concondeptodatdep}。当时已有向量机的出现,并且新一代的Fortran也一定会提供向量操作来支持向量机。不幸的是,由于之前版本的Fortran中不支持向量操作,所以相当大的一部分旧有的Fortran程序无法充分利用处理器的新特性,这就意味着,自动向量化的研究势在必行。当时Rice大学的人正在开发一个翻译程序,称为并行Fortran转换器(Parallel Fortran Converter,简称PFC),它能将Fortran 66或者Fortran 77代码翻译成新一代Fortran的代码,并且通过自动向量化来充分利用向量操作。当时的编译器进行程序变换还是基于数据依赖,分支语句产生的控制依赖不能非常好的融入到当时的体系中去。与此同时,分支语句的存在也会使得自动向量化无法进行。由于这两个原因,Allen就提出了将程序中的所有分支语句移除而转换成谓词执行的一些算法。Allen的算法的目的比较看重向量化,这算法是直接在源代码上进行操作的,它可以将所有的前向分支跟出口分支消除。但是Allen的算法没有考虑到如何最小化谓词的分配等的问题。尽管Allen的算法可以将分支转换为谓词执行,但是编译器要想判断是否可以进行向量化,必须先对代码进行if-conversion,这样就将所有的控制依赖转化为数据依赖,然后才能使用基于数据依赖的算法判断是否能够进行向量化,一旦无法向量化,不单单是if-conversion执行的时间被浪费掉,而且if-conversion会对代码的执行效率产生影响\ref{harm_performance},糟糕的是,if-conversion已经将代码变得面目全非无法还原。

1987年,Ferrante等人提出了程序依赖图(PDG)的概念\cite{ferrante1987prodepgraitsuseopt}。这是一个非常重要的工作,因为PDG是一个新的框架,在这个框架下,数据依赖跟控制依赖被统一地进行处理,一切旧有的算法都可以高效地纳入新的框架下。由于不再存在之前的编译器无法处理控制依赖的问题,也就没有必要再利用if-conversion将控制依赖转化为数据依赖了。同时这个框架对if-conversion的发展也起到了承上启下的作用,后来对if-conversion的研究都是在这个框架下的。

虽然不再需要用if-conversion将控制依赖转化为数据依赖,但是if-conversion并没还有因此失去它的意义。正如上一小节所说,程序中的分支语句已经称为高性能体系结构中性能提升的主要障碍,而if-conversion恰好可以用谓词执行来代替分支,进而提高效率。只不过问题的核心已经不是像Allen所说的那样,将控制依赖转化为数据依赖,也不是向量化,而是减少由于分支语句的存在对程序性能造成的影响,这也对if-conversion之后的代码的性能提出了要求。

