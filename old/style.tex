%document and page
\documentclass[11pt,a4paper]{article}
\usepackage[top=1.2in,bottom=1.2in,left=1.2in,right=1.2in]{geometry}

%加入了一些针对XeTeX的改进并且加入了 \XeTeX 命令来输入漂亮的XeTeX logo
\usepackage{xltxtra}
%启用一些LaTeX中的功能
\usepackage{xunicode}

%%%% fontspec 宏包 %%%%
\usepackage{fontspec}           
% 指定字体
\setmainfont[BoldFont=WenQuanYi Zen Hei]{FZSongS-Extended}
\setsansfont[BoldFont=WenQuanYi Zen Hei]{FZKai-Z03}
\setmonofont{WenQuanYi Zen Hei Mono}

%l10n of Chinese
\usepackage{indentfirst}
\setlength{\parindent}{2em}
\XeTeXlinebreaklocale "zh"
\XeTeXlinebreakskip = 0pt plus 1pt minus 0.1pt

%代码相关
\usepackage{listings} %插入代码
\usepackage{xcolor} %代码高亮
\lstset{
	numbers=left, %设置行号位置
	numberstyle=\tiny, %设置行号大小
	keywordstyle=\color{blue}, %设置关键字颜色
	commentstyle=\color[cmyk]{1,0,1,0}, %设置注释颜色
	frame=single, %设置边框格式
	escapeinside=``, %逃逸字符(1左面的键),用于显示中文
	breaklines, %自动折行
	extendedchars=false, %解决代码跨页时,章节标题,页眉等汉字不显示的问题
	xleftmargin=2em,xrightmargin=2em, aboveskip=1em, %设置边距
	tabsize=4, %设置tab空格数
	showspaces=false %不显示空格
}

%算法
\usepackage[linesnumbered,ruled]{algorithm2e}
\SetAlFnt{\small}
\SetAlCapFnt{\small}
\SetAlCapNameFnt{\small}
\SetAlCapHSkip{0pt}
\IncMargin{-\parindent}

%title and author
\def\mytitle{if-conversion研究近况及其在LLVM上的实现}
\def\myauthor{高翔}
\title{\mytitle}
\author{\myauthor}

%hyperref
\usepackage{hyperref}
\hypersetup{
	pdftitle={\mytitle},
	pdfauthor={\myauthor},
	pdfsubject={if-conversion},
	pdfkeywords={if-conversion, 分支消除, 编译器},
	bookmarksopen=true
} 
