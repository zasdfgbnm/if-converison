\subsection{RK算法}

CD函数并非是一个单值函数,不同的x可能对应相同的$CD\left(x\right)$。由于具有相同控制依赖的不同节点的执行条件相同,所以将节点转化为谓词执行以后,他们的谓词也应该相同,这样就可以为CD函数的值域中的每个元素分配一个与之一一对应的谓词p,谓词跟值域中元素的集合的对应关系记为K函数。显然,K函数是个单值函数,它的逆$K^{-1}$必然存在,记函数$R=CD\cdot K^{-1}$。可以看出,这么做将CD函数分解为R函数跟K函数:$CD=R\cdot K$。其中R函数负责将每个节点x与谓词p对应起来,而K则将每个谓词跟其对应的控制依赖对应起来。通过R函数,具有相同控制依赖(因而具有相同执行条件)的不同节点被映射到同一个谓词,也就是说,R函数为每个节点实现的谓词分配,节点x对应的谓词$p=R\left(x\right)$。至于K函数,则将每个谓词跟其控制依赖对应起来,这也为谓词的定义提供了参考。一种不错的谓词定义策略是,对于每一个谓词,在其对应的依赖,的相应节点末尾插入相应的赋值语句。例如$K\left(p\right)=\left\{y,-z\right\}$,则在y与z节点插入p的定义语句。插入的时候,如果对该点依赖值为正,则谓词的值就是分支条件,如果为赋,作为谓词的值就是该点分支条件的非。例如上面例子中,在y点插入$p=t_y$,而在z点插入$p=\neg t_z$。R函数以及K函数的计算算法如\fref{alg:ComputeRK}所示。

\begin{algorithm}[H]
	\label{alg:ComputeRK}
	\caption{ComputeRK}
	\KwIn{CD是控制依赖函数,N是G中节点的集合}
	\KwOut{R函数以及K函数}
	p = 1\tcc*{谓词命名从1开始顺序命名}
	\For{$x\in N$}{
		$t = CD\left(x\right)$\;
		\eIf{$t\in K$}{
			$R\left(x\right)=q$,其中q为使得$K\left(q\right)=t$的谓词\;
		}{
			$K\left(p\right)=t$\;
			$R\left(x\right)=p++$\;
		}
	}
\end{algorithm}
计算好R函数以及K函数,并按照R函数跟K函数进行谓词分配与定义以后,只要对控制流图进行拓扑排序即可得到转换后的代码。if-conversion的算法如\fref{alg:ifcvt}。

\begin{algorithm}[H]
	\label{alg:ifcvt}
	\caption{If-Conversion}
	\KwIn{G是控制流图,N是G中节点的集合,E是G中边的集合}
	\KwOut{转换后的代码}
	计算CD函数\;
	计算RK函数\;
	\For{$x\in N$}{
		$p=R\left(x\right)$\;
		\If{$K\left(p\right)\neq\varnothing$}{
			给x分配谓词$p$\;
		}
	}
	\For{每个谓词p}{
		\For{$\pm y\in K\left(p\right)$}{
			若是$+y$则在y尾部插入$p=t_y$\;
			若是$-y$则在y尾部插入$p=\neg t_y$;
		}
	}
	对G进行拓扑排序,删掉分支语句,并对已分配谓词的基本块进行谓词化\;
\end{algorithm}

%例如,如下控制流图:
%举个例子说明一下