\section{LLVM上if-conversion的实现}

尽管if-conversion的研究成果丰硕,但是不幸的是这些成果并没有在LLVM中得到应用。在LLVM 3.4中,if-conversion的实现如下:

LLVM中的if-conversion的实现非常简单,并没用所谓的RK算法,而是用的一个相当简单的算法。在lib\/CodeGen目录中有两个文件与if-conversion相关,一个是IfConversion.cpp,另一个是EarlyIfConversion.cpp。其中,前者作用于机器码,由AnalyzeBranch调用,后者则作用于SSA格式的IR,这是给乱序处理器(out-of-order processor)使用,乱序处理器的谓词执行能力比较弱。LLVM中if-conversion的机制很简单,它只考虑结构化编程,则分支可以解成三种类型:简单分支,三角分支,菱形分支。三种分支的控制流图分别如下所示:

\begin{tabular}{|c|c|c|}
\hline
菱形分支 & 三角形分支 & 简单分支\\
{\xymatrix{
&A\ar[ld]_-{T}\ar[rd]^{F}&\\
B\ar[rd]& &C\ar[ld]\\
&D&
}} & {\xymatrix{
&A\ar[ld]_-{T}\ar[dd]^{F}&\\
B\ar[rd]& &\\
&C&
}} & {\xymatrix{
&A\ar[ld]_-{T}\ar[d]^{F}&\\
exit&B\ar[d]&\\
&C&
}}\\
\hline
\end{tabular}

对于这三种类型的分支,都可以将分支条件作为谓词,处理起来比较简单。