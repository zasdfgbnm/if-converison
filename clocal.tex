%段落缩进
\usepackage{indentfirst}
\setlength{\parindent}{20pt}
%中文设置
\usepackage[cm-default]{fontspec}
\usepackage{xeCJK}
\defaultfontfeatures{Mapping=tex-text}
% 中文断行
\XeTeXlinebreaklocale "zh"
\XeTeXlinebreakskip = 0pt plus 1pt minus 0.1pt
% 行距
\linespread{1.25} 
%将系统字体名映射为逻辑字体名称
\newcommand\fontnameroman{LMRoman10}
\newcommand\fontnamesans{LMSans10}
\newcommand\fontnamesong{宋体}
\newcommand\fontnamehei{黑体}
\newcommand\fontnamekai{华文楷体}
\setmainfont{LMRoman10}
\setCJKmainfont{华文楷体}
\setsansfont{LMSans10}
%楷体
\newfontinstance\KAI {\fontnamekai}
\newcommand{\kai}[1]{{\KAI#1}}
%黑体
%\newfontinstance\HEI{\fontnamehei}
%\newcommand{\hei}[1]{{\HEI#1}}
\setCJKfamilyfont{HEI}{黑体}
\newcommand\hei[1]{{\CJKfamily{HEI}#1}}
%英文
\newfontinstance\ENF{\fontnameroman}
\newcommand{\en}[1]{\,{\ENF#1}\,}
\newfontinstance\ENSF{\fontnamesans}
\newcommand{\ensf}[1]{\,{\ENSF#1}\,}
%减少itemize的行间距
\usepackage{paralist} 
\let\itemize\compactitem 
\let\enditemize\endcompactitem 
\let\enumerate\compactenum 
\let\endenumerate\endcompactenum 
\let\description\compactdesc 
\let\enddescription\endcompactdesc
%作者之间的连接
\renewcommand\Authand{ \quad }
%目录
\renewcommand\contentsname{\hei{目 录}} 
\if\hasalgo
\floatname{algorithm}{\hei{算 法}} 
\fi
\renewcommand\figurename{\hei{图}} 
\renewcommand\tablename{\hei{表}} 
\renewcommand{\refname}{\hei{参 考}}
\renewcommand{\lstlistingname}{\hei{代码}} %% 重命名Listings标题头
%
