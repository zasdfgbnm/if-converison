
%%%%%%%%%%%%%%%%%%%%%%%%%%%%%%%%%%%%%%%%%%%%%%%%%%%%%%%%%%%%%%%%%%%%%%%%%%%%%
%  Including standard latex packages
%%%%%%%%%%%%%%%%%%%%%%%%%%%%%%%%%%%%%%%%%%%%%%%%%%%%%%%%%%%%%%%%%%%%%%%%%%%%%
%

\usepackage{stmaryrd}            %more math symbols
\usepackage{amsmath}             %ams math
\usepackage{amsfonts}            %ams fonts
\usepackage{amssymb}             %amsmath
\usepackage{mathrsfs}            % \mathscr
\usepackage{latexsym}            %latex symbols
\usepackage{graphicx}            %grahpicx
\usepackage{tabularx}
\usepackage{subfigure}
\usepackage{multirow}
\usepackage{multicol}
\usepackage{cite}                % sort citation numbers
\usepackage{url}
\usepackage{authblk}
\usepackage{framed}



%fancyref
\usepackage[plain]{fancyref}
\newcommand*{\fancyrefapplabelprefix}{app}
\newcommand*{\Frefappname}{Appendix}
\frefformat{plain}{\fancyrefapplabelprefix}{\Frefappname\fancyrefdefaultspacing#1}
\newcommand*{\fancyrefalglabelprefix}{alg}
\newcommand*{\Frefalgname}{Algorithm}
\frefformat{plain}{\fancyrefalglabelprefix}{\Frefalgname\fancyrefdefaultspacing#1}

%appendix
\usepackage[page,title,titletoc,header]{appendix}

%algorithm
%\if\hasalgo
%\usepackage{float}
%\usepackage{xspace}
%\usepackage{algorithmwh}
%\fi
\usepackage[linesnumbered,ruled]{algorithm2e}
\SetAlFnt{\small}
\SetAlCapFnt{\small}
\SetAlCapNameFnt{\small}
\SetAlCapHSkip{0pt}
\IncMargin{-\parindent}

\usepackage[all]{xy}             %various figures
%\usepackage[amsmath,thmmarks]{ntheorem}%theorems
\usepackage[final]{listings}
\usepackage[]{hyperref}
\hypersetup{
	bookmarksnumbered,%
	bookmarksopen,%
	colorlinks,%
	citecolor=blue,%
	filecolor=magenta,%
	linkcolor=blue,%
	%urlcolor=green,%
	hyperindex,%
	plainpages=false,%
	pdfstartview=FitH
}

\usepackage[usenames,dvipsnames]{color}
\if\showchange
\newcommand\bluet[1]{\textcolor{blue}{#1}}
\newcommand\redt[1]{\textcolor{red}{#1}}
\newcommand\greent[1]{\textcolor{green}{#1}}
\else
\newcommand\bluet[1]{#1}
\newcommand\redt[1]{#1}
\newcommand\greent[1]{\textcolor{green}{#1}}
\fi

\newcommand\codet[1]{\textcolor{Brown}{\texttt{#1}}}
\newlength\savewidth
\newcommand\whline{\noalign{\global\savewidth\arrayrulewidth
                            \global\arrayrulewidth 1pt}%
                   \hline
                   \noalign{\global\arrayrulewidth\savewidth}}
%%%%%%%%%%%%%%%%%%%%%%%%%%%%%%%%%%%%%%%%%%%%%%%%%%%%%%%%%%%%%%%%%%%%%%%%%%%%%
%  Including all the local shared latex packages and macros
%%%%%%%%%%%%%%%%%%%%%%%%%%%%%%%%%%%%%%%%%%%%%%%%%%%%%%%%%%%%%%%%%%%%%%%%%%%%%
\renewcommand{\ttdefault}{cmtt}
\let\cmttdfl\ttdefault          %what for?
\let\cmsfdfl\sfdefault          %what for?

\newcommand\etal{\emph{et al.\ }}
\newcommand\eg{\emph{e.g.,\ }}
\newcommand\ie{\emph{i.e.,\ }}
\newcommand\etc{\emph{etc.\ }}
%%%%%%%%%%%%%%%%%%%%%%%%%%%%%%%%%%%%%%%%%%%%%%%%%%%%%%%%%%%%%%%%%%%%%%%%%%%%%
% THE START OF THE MAIN DOCUMENT
%%%%%%%%%%%%%%%%%%%%%%%%%%%%%%%%%%%%%%%%%%%%%%%%%%%%%%%%%%%%%%%%%%%%%%%%%%%%%
\lstset{language=C,%
	basicstyle=\footnotesize\ttfamily,%\small,%
	columns=fixed,%
	keepspaces=true,%
	lineskip=2pt,%
	%boxpos=c,%
	tabsize=4,%
	frame=single,%
	%numbers=left,%
	xleftmargin=0em,xrightmargin=0em, aboveskip=1em,
%	escapeinside=``,
	breaklines=true,%这条命令可以让LaTeX自动将长的代码行换行排版
	extendedchars=false %这一条命令可以解决代码跨页时,章节标题,页眉等汉字>不显示的问题
}

% Choose abbreviated or long-version alternatives in paper
\long\def\abbr#1#2{#1}                  % abbreviated version
%\long\def\abbr#1#2{#2}                 % long version

% Choose abbreviations or long names/titles in bibliography
\def\bibbrev#1#2{#1}                    % short version
%\def\bibbrev#1#2{#2}                   % long version
%\def\bibbrev#1#2{\abbr{#1}{#2}}                % follow abbr macro

% Abbreviated or full citation lists: \abcite{basic}{others}
\newcommand{\abcite}[2]{\abbr{\cite{#1}}{\cite{#1,#2}}}

% Conference abbreviations: \bibconf[Nth]{SOSP}{Symposium on ...}
\newcommand{\bibconf}[3][]{#1 \bibbrev{#2}{#3 (#2)}}

%代码相关
\usepackage{listings} %插入代码
\usepackage{xcolor} %代码高亮
\lstset{
	numbers=left, %设置行号位置
	numberstyle=\tiny, %设置行号大小
	keywordstyle=\color{blue}, %设置关键字颜色
	commentstyle=\color[cmyk]{1,0,1,0}, %设置注释颜色
	frame=single, %设置边框格式
	escapeinside=``, %逃逸字符(1左面的键),用于显示中文
	breaklines, %自动折行
	extendedchars=false, %解决代码跨页时,章节标题,页眉等汉字不显示的问题
	xleftmargin=2em,xrightmargin=2em, aboveskip=1em, %设置边距
	tabsize=4, %设置tab空格数
	showspaces=false %不显示空格
}

%tikz%
\usepackage{tikz}
\usetikzlibrary{shapes,arrows}

% Define block styles
\tikzstyle{decision} = [diamond, draw, fill=blue!20, 
    text width=4.5em, text badly centered, node distance=3cm, inner sep=0pt]
\tikzstyle{block} = [rectangle, draw, fill=blue!20, 
    text width=5em, text centered, rounded corners, minimum height=4em]
\tikzstyle{line} = [draw, -latex']
\tikzstyle{cloud} = [draw, ellipse,fill=red!20, node distance=3cm,
    minimum height=2em]