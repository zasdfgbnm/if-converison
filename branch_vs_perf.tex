\section{分支与性能的关系}

\subsection{分支对性能的阻碍}

在现代的高性能处理器中,分支指令是一个代价昂贵的指令\cite{Mahlke1994}。深度流水以及多发射极大的提高了处理器的性能,但是频繁出现的分支导致的控制流转移会中断指令流水线的连续性,从而降低性能。为了解决分支语句对性能的影响,现代处理器往往配置分支预测机制,即在条件分支执行前,处理器对分支方向进行预测,并按照预测的方向进行取指执行。不幸的是,分支预测经常出错,一旦预测失败就会造成非常严重的性能损失。研究显示,错误的分支预测会使得性能降低2-10倍\cite{Smith1989}\cite{Wall1991}\cite{1991}。

分支预测失败对性能的影响主要有三个方面。首先,处理器会在预测的分支方向取出大量指令执行,一旦分支预测失败,所有执行的结果都必行被丢弃。这就意味着,预测失败时,处理器浪费了大量的指令槽来执行无用的指令。浪费的指令槽的数目与处理器的发射宽度有关,处理器发射宽度越高,浪费的指令槽的数目也就越多。其次,分支预测失败以后,处理器必须撤销预测执行对处理器的影响,处理器必选让流水线的排空,同时还要无效化处理器缓存,以避免其内容被更新到处理器状态。再次,当发现预测失败的时候,处理器必须恢复到正确的分支进行执行,这就需要计算正确的指令地址,以及初始化正确方向的取指过程。另外,由于由于分支语句大量存在于程序中,超标量处理器必须要能够在一个周期内执行多个分支。假设指令流中包含25\%的分支,那么一个8发射的超标量处理器必须具有每个周期执行至少2个分支的能力。一个周期执行多个分支增加了流水线的复杂度。在一个高发射率的处理器中,重复算术功能单元要比预测并执行多个分支要容易得多。

除了分支预测失败造成的处罚以外,分支语句会将程序分割为若干小的基本块,指令调度无法在基本块之间进行,从而影响性能。


\subsection{谓词执行对性能的改善}

为了解决造成的种种问题,现代处理器提供谓词执行的机制。谓词执行指的是指令的条件执行,它给指令增加一个布尔操作数,这个布尔操作数称为谓词。如果谓词的值为真,那么这个指令被正常地执行,而如果这个谓词的值为假,则这个指令就被当成空指令来执行。

有了处理器的支持,编译器就可以通过if-conversion来将分支转化为谓词执行。通过分支的移除,可以消除分支预测失败的惩罚,也可以避免让宽发射处理器在一个周期内处理多条分支从而提高了性能。

谓词执行的支持已经被非常广泛地应用在数值型程序以及非数值型程序的调度中\cite{ScottA.Mahlke1992}。对于数值型的代码,使用软件流水调度对多个循环迭代进行重叠执行可以在超标量以及VLIW机器上获得高性能\cite{1989a}\cite{JosephP.Brutt}。分支的移除使得软件流水线能够进行更紧凑的调度以及更简短的代码展开。对于非数值型的程序,谓词执行使得决定树调度在深流水以及多指令发射的处理器上获得更高的性能\cite{Hsu1986}。谓词执行允许多个控制路径上指令的并发执行,并且使指令在其依赖的分支被解析之前的执行成为可能。

\subsection{分支消除对性能的影响}\label{ifcvt_harm_performance}

虽然适当地应用if-conversion可以对性能进行提升,但是过多地进行if-conversion也会对性能造成不利的影响,并且不多地进行if-conversion也不一定改善性能。\cite{ScottA.Mahlke1992}\cite{Tian2010}。

首先,if-conversion会把一个区域内的所有的执行路径合并成一条较长的路径,所以,每次基本块被执行的时候,所有指令都必须被执行。如果不同的路径具有相近的频率跟大小,这个时候代码的执行是高效的。但是不同路径的大小以及频率往往是不同的,这就导致程序每次都要浪费大量的时间用来执行不频繁的分支或者长的分支,这对性能是不利的影响。同时,一些分支中的函数调用或者未解析的内存访问也会限制转换后的大基本块的优化以及调度。

其次,预测执行跟谓词执行并不能很好地符合到一起。预测执行指的是在指令真正需要被执行之前将指令执行,对于谓词指令来说,预测执行表示的是在其谓词计算之前对指令的执行。预测执行是超标量以及VLIW处理器中非常重要的指令级并行的来源,因为它能使得长延迟指令在调度中非常早地被初始化。

再次,由于多个基本块被合并成了一个大的基本块,他们将共享处理器资源,这就造成处理器资源的争用,这会增大分配的寄存器的数量。如果寄存器的分配对谓词不敏感,这将会对寄存器的分配产生压力\cite{Quinones2006}\cite{Quinones2007}。