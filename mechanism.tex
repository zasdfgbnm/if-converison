\section{if-conversion的运行机制}

控制依赖的概念最早由Ferrante等人提出\cite{ferrante1987prodepgraitsuseopt},RK算法由Park跟Schlansker提出\cite{JosephC.H.Park1991},Hyperblock的概念由Mahlke等人给出\cite{ScottA.Mahlke1992a},逆向if-conversion则是Warter等人的研究成果\cite{Warter1993}。

\subsection{控制依赖}

\begin{definition}[控制流图(control flow graph)]
控制流图是一个有向图,它有唯一的入口节点START以及唯一的出口节点STOP,图中每一个节点最多有两个后继。对于那些有两个后继的节点,它的两条出边分别被赋予属性T(真)跟F(假)。对于图中的每个节点N,都存在从START到N,以及从N到STOP的路径。
\end{definition}

\begin{definition}[后控制(post-dominate)]
设V与W是控制流图G中的节点,如果从V到STOP的每条路径(起始位置的V并不计算在路径中)都包含W,则称V被W后控制,或者W后控制V,记为$W \pdom V$。
\end{definition}

\begin{definition}[直接后控制(immediate post-dominate)]
设X,Y以及Z是控制流图G中的节点,如果$Y\pdom X$,并且对于任意的满足$Z\pdom X$以及$Z\neq Y$的节点Z,都有$Z\pdom Y$,则称Y直接后控制X,或者X被Y直接后控制,记为$Y \ipdom X$
\end{definition}

\begin{definition}[pdom函数]
设N是节点的集合,pdom是一个映射,它将N中的节点x映射为所有后控制x的节点的集合,即$pdom: N\to 2^N$。pdom的定义的数学表述为:$pdom\left(x\right):=\left\{y\in N: y \pdom x\right\}$
\end{definition}

\begin{definition}[ipdom函数]
设N是节点的集合,ipdom是一个映射,它将N中的节点x映射为x直接后控制的节点的节点,即$pdom: N\to N$。ipdom的定义的数学表述为:$ipdom\left(x\right):=y\in N,\text{其中}y \ipdom x$
\end{definition}

\begin{definition}[后控制节点树]
直接后控制关系构成一个树,称为后控制节点树。树的节点为控制流图G中的节点,若$y\ipdom x$,则y是x的双亲节点。显然,求$ipdom\left(x\right)$即为求x在树中的双亲节点,求$pdom\left(x\right)$即为求x在树中的所有祖先节点的集合。
\end{definition}
计算图G的后控制节点树的算法见\fref{app:algo_ipdom_tree}

\begin{definition}[控制依赖(control dependent)]
设G为控制流图,X以及Y是图中节点,称Y控制依赖于X当且仅当:
\begin{enumerate}
\item 存在从X到Y的路径P,使得Y后控制P上除了X与Y的所有节点。
\item X不被Y后控制
\end{enumerate}
\end{definition}
从定义可以看出,如果Y控制依赖与X,则那么X必然有两个出口,其中一条出口导致Y必然被执行,另一条出口则导致Y可能不被执行。

\begin{definition}[CD函数]
设N是节点的集合,C是控制依赖的集合。CD函数是一个映射,它将N中的节点x映射为x的所有控制依赖的集合,即$CD:N\to 2^C$。若将C中的元素$c\in C$表示为$\pm y$,其中$+y$表示y的true边,$-y$表示y的false边,则CD函数的定义可以为:$CD\left(x\right):=\left\{\pm y:x\text{控制依赖于}\pm y\right\}$
\end{definition}
CD函数的计算算法如\fref{alg:ComputeCD}

\begin{algorithm}[H]
	\label{alg:ComputeCD}
	\caption{ComputeCD(G)}
	\KwIn{G是控制流图,N是G中节点的集合,E是G中边的集合}
	\KwOut{CD函数}
	引入一个新的边$\left[Start,Stop,false\right]$\;
	计算函数$pdom\left(x\right)$以及$ipdom\left(x\right)$\;
	\For{每个$\left[x,y,label\right]\in E$并且$y\notin pdom\left(x\right)$}{
		Lub = ipdom(x)\;
		\If{$\neg label$}{$x= -x$\;}
		$t=y$\;
		\While{$t\neq Lub$}{
			$CD\left(t\right)=CD\left(t\right)\cup\left\{x\right\}$\;
			$t=ipdom\left(t\right)$\;
		}
	}
	移除$\left[Start,Stop,false\right]$\;
\end{algorithm}

\subsection{RK算法}
\subsection{Hyperblock}
\subsection{逆向if-conversion}