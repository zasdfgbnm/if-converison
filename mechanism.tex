\section{if-conversion的运行机制}

控制依赖的概念最早由Ferrante等人提出\cite{ferrante1987prodepgraitsuseopt},后控制节点树的计算算法由Lengauer以及Tarjan给出\cite{Lengauer1979},RK算法由Park跟Schlansker提出\cite{JosephC.H.Park1991},Hyperblock的概念由Mahlke等人给出\cite{ScottA.Mahlke1992a},逆向if-conversion则是Warter等人的研究成果\cite{Warter1993}。

\subsection{控制依赖}

\begin{definition}[控制流图(control flow graph)]
控制流图是一个有向图,它有唯一的入口节点START以及唯一的出口节点STOP,图中每一个节点最多有两个后继。对于那些有两个后继的节点,它的两条出边分别被赋予属性T(真)跟F(假)。对于图中的每个节点N,都存在从START到N,以及从N到STOP的路径。
\end{definition}

\begin{definition}[后控制(post-dominate)]
如果在控制流图中,从V到STOP的每条路径(起始位置的V并不计算在路径中)都包含W,则称V被W后控制,或者W后控制V。
\end{definition}

\begin{definition}[控制依赖(control dependent)]
设G为控制流图,X以及Y是图中节点,称Y控制依赖于X当且仅当:
\begin{enumerate}
\item 存在从X到Y的路径P,使得Y后控制P上除了X与Y的所有节点。
\item X不被Y后控制
\end{enumerate}
\end{definition}
从定义可以看出,如果Y控制依赖与X,则那么X必然有两个出口,其中一条出口导致Y必然被执行,另一条出口则导致Y可能不被执行。

\subsubsection{后控制节点树的计算}
后控制节点树的计算

\subsection{RK算法}
\subsection{Hyperblock}
\subsection{逆向if-conversion}