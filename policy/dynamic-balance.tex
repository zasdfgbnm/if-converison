\subsection{动态平衡框架}

动态if-conversion的概念是Hazelwood在2000年提出的\cite{Hazelwood00alightweight},这里面的“动态”的意思是说,根据程序的运行情况来实时地对代码进行if-conversion。相比较动态if-conversion,之前的August的策略叫做静态if-conversion,因为所有的转换工作以及是否转换的决定已经在编译的时候就确定了。动态if-conversion的优点是,静态分析很难得出准确的分支预测失败概率,并且这些参数往往会随着程序的运行而改变,动态if-conversion可以随时调整程序到最佳状态以保证程序随时都能高效运行。Hazelwood的做法是,程序运行的过程中,动态监视程序的某些运行时参数,不如说分支命中率,分支预测失败惩罚等等,当这些参数达到某个阈值的时候,即对程序进行if-conversion,将相应的分支转换成谓词执行,当这些参数达到另一个阈值的时候,进行逆向if-conversion这样通过动态地对程序进行分支与谓词执行之间的转换,达到性能最优。