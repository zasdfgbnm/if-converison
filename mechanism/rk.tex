\subsection{RK算法}

CD函数并非是一个单值函数,不同的x可能对应相同的$CD\left(x\right)$。由于具有相同控制依赖的不同节点的执行条件相同,所以将节点转化为谓词执行以后,他们的谓词也应该相同,这样就可以为CD函数的值域中的每个元素分配一个与之一一对应的谓词p,谓词跟值域中元素的集合的对应关系记为K函数。显然,K函数是个单值函数,它的逆$K^{-1}$必然存在,记函数$R=CD\cdot K^{-1}$。可以看出,这么做将CD函数分解为R函数跟K函数:$CD=R\cdot K$。其中R函数负责将每个节点x与谓词p对应起来,而K则将每个谓词跟其对应的控制依赖对应起来。通过R函数,具有相同控制依赖(因而具有相同执行条件)的不同节点被映射到同一个谓词,也就是说,R函数为每个节点实现的谓词分配,节点x对应的谓词$p=R\left(x\right)$。至于K函数,则将每个谓词跟其控制依赖对应起来,这也为谓词的定义提供了参考。一种不错的谓词定义策略是,对于每一个谓词,在其对应的依赖,的相应节点末尾插入相应的赋值语句。例如$K\left(p\right)=\left\{y,-z\right\}$,则在y与z节点插入p的定义语句。插入的时候,如果对该点依赖值为正,则谓词的值就是分支条件,如果为赋,作为谓词的值就是该点分支条件的非。例如上面例子中,在y点插入$p=t_y$,而在z点插入$p=\neg t_z$。R函数以及K函数的计算算法如\fref{alg:ComputeRK}所示。

\begin{algorithm}[H]
	\label{alg:ComputeRK}
	\caption{ComputeRK}
	\KwIn{控制流图$G\left(N,E,Start\right)$,控制依赖函数CD}
	\KwOut{R函数以及K函数}
	p = 1\tcc*{谓词命名从1开始顺序命名}
	\For{$x\in N$}{
		$t = CD\left(x\right)$\;
		\eIf{$t\in K$}{
			$R\left(x\right)=q$,其中q为使得$K\left(q\right)=t$的谓词\;
		}{
			$K\left(p\right)=t$\;
			$R\left(x\right)=p++$\;
		}
	}
\end{algorithm}
计算好R函数以及K函数,并按照R函数跟K函数进行谓词分配与定义以后,只要对控制流图进行拓扑排序即可得到转换后的代码。但是此时的代码有一定的问题:存在谓词未被定义就被使用。造成这个问题的原因是因为,控制流图中的所有路径都被压缩到了一条路径上,这样就会导致某些分支的谓词未被定义。比如这样的控制流图:\\
\xymatrix{
						&*+[F]{Start}\ar[d]							\\
						&*+[F]{A}\ar[ld]_{T}\ar[rd]^{F}					\\
*+[F]{B}\ar[d]_{T}\ar[rrd]^{F}	&						&*+[F]{C}\ar[d]		\\
*+[F]{D}\ar[rd]				&						&*+[F]{E}\ar[ld]	\\
						&*+[F]{F}\ar[d]								\\
						&*+[F]{End}
}\\
D只对B有控制依赖,对A并没有控制依赖,因此D的谓词只会在B处被定义,但是如果在A处分支的到时候走的是false那一支,则B将不会被执行,也就是说D的谓词将不会被定义。

要解决这个问题最简单最直观的方法是在程序开始的地方将所有的谓词初始化为$false$,但是未被定义就被使用的谓词毕竟只是少数,这么做会引入很多不必要的浪费。聪明的做法是找到哪些谓词存在未被定义就被使用的问题。定义集合$A\left(x\right):=\left\{p\in P': \exists path\left(x,Stop\right)\text{使得}K\left(p\right)\cap path\left(x,Stop\right)=\varnothing\right\}$,其中$P'$为所有$K\left(p\right)\neq\varnothing$的谓词的集合,$path\left(x,Stop\right)$为从x到Stop的路径。显然,$A\left(Start\right)$就是应该在程序开始的时候初始化为$false$的谓词的集合。由于$A\left(x\right)$的计算比较麻烦,一种方便的做法是注意到所有满足$K\left(p\right)\cap\left\{\pm x\in E:x \pdom Head\right\}\neq\varnothing$($Head$指的是Start的唯一直接后继)的谓词都不是$A\left(Start\right)$的成员,所以可以将所有不满足上述条件的谓词全部在程序的开头初始化为$false$,这样可以在浪费比较小的处理器资源的前提下省去计算$A\left(x\right)$的麻烦。

$A\left(x\right)$的计算是通过将其转化为定义使用链问题来实现的。将所有的基本块$B_x$抽象为一个点$x$,该点使用分配到$B_x$的谓词,并且定义由$K\left(p\right)$给出的对应谓词,即对于每个$b\in N$除了$Stop$,定义$Use\left(b\right)=\left\{p\in P':p=R\left(b\right)\right\}$,$Def\left(b\right)=\left\{p\in P':\pm b \in K\left(p\right)\right\}$,规定$Use\left(Stop\right)=P'$。这样的话通过解数据流方程
\[
\left\{ \begin{array}{l}
IN\left(b\right)=Use\left(b\right)\cup\left[OUT\left(b\right)-Def\left(b\right)\right]\\
OUT\left(b\right)=\underset{s\in succ\left(b\right)}{\bigcup}IN\left(s\right)
\end{array}\right.
\]
得到$IN\left(x\right)$,即可求得$A\left(Start\right)=IN\left(Start\right)$。具体算法如\fref{alg:du}

\begin{algorithm}[H]
	\label{alg:du}
	\caption{SolveDataFlowEquations}
	\KwIn{控制流图$G\left(N,E,Start\right)$,R函数,K函数}
	\KwOut{$IN$函数,$OUT$函数}
	$N'=N-\left\{Stop\right\}$\;
	计算$Def\left(b\right)$以及$Use\left(b\right)$\;
	$IN=Use$\;
	$change=true$\;
	\While{change}{
		$change=false$\;
		\For{$b\in N'$}{
			$OUT\left(b\right)=\varnothing$\;
			\For{$s\in succ\left(b\right)$}{
				$OUT\left(b\right)=OUT\left(b\right)\cup IN\left(s\right)$\;
			}
			$old=IN\left(b\right)$\;
			$IN\left(b\right)=Use\left(b\right)\cup\left[OUT\left(b\right)-Def\left(b\right)\right]$\;
			\If{$old\neq IN\left(b\right)$}{
				$change=true$\;
			}
		}
	}
\end{algorithm}

综上所述,if-conversion的算法概括为\fref{alg:ifcvt}。

\begin{algorithm}[H]
	\label{alg:ifcvt}
	\caption{If-Conversion}
	\KwIn{G是控制流图,N是G中节点的集合,E是G中边的集合}
	\KwOut{转换后的代码}
	计算CD函数\;
	计算RK函数\;
	\For{$x\in N$}{
		$p=R\left(x\right)$\;
		\If{$K\left(p\right)\neq\varnothing$}{
			给x分配谓词$p$\;
		}
	}
	\For{每个谓词p}{
		\For{$\pm y\in K\left(p\right)$}{
			若是$+y$则在y尾部插入$p=t_y$\;
			若是$-y$则在y尾部插入$p=\neg t_y$;
		}
	}
	对G进行拓扑排序,删掉分支语句,并对已分配谓词的基本块进行谓词化\;
	计算$IN\left(b\right)$\;
	\For{$p\in IN\left(Start\right)$}{
		在程序的开头插入“$p=false$”\;
	}
\end{algorithm}

使用RK算法进行if-conversion的示例如下:
如下控制流图,基本块的标签表示为$B_i\left(t_i\right)\&p_i$。图中$B_i$为基本块名称;$t_i$表示该基本块末尾会按照条件$t_i$分支,如果基本块不进行分支,则$\left(t_i\right)$不写;而$p_i$则表示谓词,如果基本块被无条件执行,则$p_i$也不写。

转换前的控制流图为:\\
\newcommand{\blockB}[2]{\parbox{5em}{\centering $B_{#1}$ \\ 代码段{#1};\\#2}}
\newcommand{\blockBT}[2]{\parbox{5em}{\centering $B_{#1}\left(t_{#1}\right)$ \\ 代码段{#1};\\#2}}
\begin{tikzpicture}[node distance = 2cm, auto]
	% Place nodes
	\node [cloud] (start) {Start};
	\node [block, below of=start] (B1) {\blockBT{1}{}};
	\node [block, below left of=B1, node distance=3cm] (B2) {\blockBT{2}{}};
	\node [block, below right of=B1, node distance=3cm] (B3) {\blockBT{3}{}};
	\node [block, below of=B2] (B4) {\blockB{4}{}};
	\node [block, below of=B3] (B5) {\blockB{5}{}};
	\node [block, below right of=B4, node distance=3cm] (B6) {\blockB{6}{}};
	\node [block, below of=B6] (B7) {\blockB{7}{}};
	\node [cloud, below of=B7, node distance=2cm] (stop) {Stop};
	% Draw edges
	\path [line] (start) -- (B1);
	\path [line] (B1) -- node [above] {T} (B2);
	\path [line] (B1) -- node {F} (B3);
	\path [line] (B2) -- node {T} (B4);
	\path [line] (B2) -- node {F} (B5);
	\path [line] (B3) -- node {T} (B5);
	\path [line] (B3) to[bend left=70] node {F} (B7);
	\path [line] (B4) -- (B6);
	\path [line] (B5) -- (B6);
	\path [line] (B6) -- (B7);
	\path [line] (B7) -- (stop);
\end{tikzpicture}\\
计算控制依赖得:\\
\begin{tabular}{|c|c|c|c|c|c|c|c|}
\hline
基本块 & 1 & 2 & 3 & 4 & 5 & 6 & 7 \\
\hline
$CD\left(x\right)$ & $\varnothing$ & $\left\{1\right\}$ & $\left\{-1\right\}$ & $\left\{2\right\}$ & $\left\{-2,3\right\}$ & $\left\{1,3\right\}$ & $\varnothing$ \\
\hline
\end{tabular}\\
将控制依赖分解为R函数跟K函数:\\
\begin{tabular}{|c|c|c|c|c|c|c|c|}
\hline
基本块 & 1 & 2 & 3 & 4 & 5 & 6 & 7 \\
\hline
$R\left(x\right)$ & $p_1$ & $p_2$ & $p_5$ & $p_4$ & $p_3$ & $p_6$ & $p_1$ \\
\hline
\end{tabular}\\
\begin{tabular}{|c|c|c|c|c|c|c|}
\hline
谓词 & $p_1$ & $p_2$ & $p_3$ & $p_4$ & $p_5$ & $p_6$ \\
\hline
$K\left(x\right)$ & $\varnothing$ & $\left\{1\right\}$ & $\left\{-2,3\right\}$ & $\left\{2\right\}$ & $\left\{-1\right\}$ & $\left\{1,3\right\}$\\
\hline
\end{tabular}\\
按照RK函数进行谓词分配与定义:\\
\newcommand{\blockBP}[3]{\parbox{5em}{\centering $B_{#1}\&p_{#2}$ \\ 代码段{#1};\\#3}}
\newcommand{\blockBTP}[3]{\parbox{5em}{\centering $B_{#1}\left(t_{#1}\right)\&p_{#2}$ \\ 代码段{#1};\\#3}}
\begin{tikzpicture}[node distance = 2cm, auto]
	% Place nodes
	\node [cloud] (start) {Start};
	\node [block, below of=start, node distance=2.5cm] (B1) {\blockBT{1}{$p_2=t_1$;\\$p_5=\neg t_1$;\\$p_6=t_1$;}};
	\node [block, below left of=B1, node distance=4cm] (B2) {\blockBTP{2}{2}{$p_3=\neg t_2$;\\$p_4=t_2$;}};
	\node [block, below right of=B1, node distance=4cm] (B3) {\blockBTP{3}{5}{$p_3=t_3$;\\$p_6=t_3$;}};
	\node [block, below of=B2, node distance=3cm] (B4) {\blockBP{4}{4}{}};
	\node [block, below of=B3, node distance=3cm] (B5) {\blockBP{5}{3}{}};
	\node [block, below right of=B4, node distance=4cm] (B6) {\blockBP{6}{6}{}};
	\node [block, below of=B6] (B7) {\blockB{7}{}};
	\node [cloud, below of=B7, node distance=2cm] (stop) {Stop};
	% Draw edges
	\path [line] (start) -- (B1);
	\path [line] (B1) -- node [above] {T} (B2);
	\path [line] (B1) -- node {F} (B3);
	\path [line] (B2) -- node {T} (B4);
	\path [line] (B2) -- node {F} (B5);
	\path [line] (B3) -- node {T} (B5);
	\path [line] (B3) to[bend left=70] node {F} (B7);
	\path [line] (B4) -- (B6);
	\path [line] (B5) -- (B6);
	\path [line] (B6) -- (B7);
	\path [line] (B7) -- (stop);
\end{tikzpicture}\\
进行拓扑排序即得排序以后的代码:\\
\fbox{\parbox{10em}{
代码段1;\\
$p_2=t_1$;\\
$p_5=\neg t_1$;\\
$p_6=t_1$;\\
if($p_2$) 代码段2;\\
if($p_2$) $p_3=\neg t_2$;\\
if($p_2$) $p_4=t_2$;\\
if($p_5$) 代码段3;\\
if($p_5$) $p_3=t_3$;\\
if($p_5$) $p_6=t_3$;\\
if($p_4$) 代码段4;\\
if($p_3$) 代码段5;\\
if($p_6$) 代码段6;\\
代码段7;
}}\\
计算得$IN\left(Start\right)=\left\{p_4\right\}$,于是最终代码为:\\
\fbox{\parbox{10em}{
$p_4=false$;\\
代码段1;\\
$p_2=t_1$;\\
$p_5=\neg t_1$;\\
$p_6=t_1$;\\
if($p_2$) 代码段2;\\
if($p_2$) $p_3=\neg t_2$;\\
if($p_2$) $p_4=t_2$;\\
if($p_5$) 代码段3;\\
if($p_5$) $p_3=t_3$;\\
if($p_5$) $p_6=t_3$;\\
if($p_4$) 代码段4;\\
if($p_3$) 代码段5;\\
if($p_6$) 代码段6;\\
代码段7;
}}\\