%document and page
\documentclass[11pt,a4paper]{article}
\usepackage[top=1.2in,bottom=1.2in,left=1.2in,right=1.2in]{geometry}

%l10n of Chinese
\usepackage{fontspec}
\setmainfont[BoldFont=WenQuanYi Zen Hei]{FZSongS-Extended}
\setsansfont[BoldFont=WenQuanYi Zen Hei]{FZKai-Z03}
\setmonofont{WenQuanYi Zen Hei Mono}
\usepackage{indentfirst}
\setlength{\parindent}{2em}
\XeTeXlinebreaklocale "zh"
\XeTeXlinebreakskip = 0pt plus 1pt minus 0.1pt

%title and author
\def\mytitle{if-conversion研究近况及其在LLVM上的实现}
\def\myauthor{高翔}
\title{\mytitle}
\author{\myauthor}

%hyperref
\usepackage{hyperref}
\hypersetup{
 pdftitle={\mytitle},
 pdfauthor={\myauthor},
 pdfsubject={if-conversion},
 pdfkeywords={if-conversion, 分支消除, 编译器}
}


%document begins here
\begin{document}

\maketitle

\begin{abstract}
高性能的体系结构的发展对编译器对程序的优化提出了更高的要求。
一个先进的编译器,必然会面临这样一个问题:何时可以调换指令的顺序而保持程序的含义不变。
编译器在解决上述问题的时候,最常用的策略是基于依赖的。
可以通过指令对相同存储单元的访问情况来定义数据依赖。
然而,程序中往往存在分支语句,分支语句会使得依赖关系变得复杂。
编译器如果没有意识到分支语句对依赖的影响,将会产生不正确的代码。
对分支语句,有两种处理方法,第一种是if转换(if-conversion),另一种是引入新的依赖类型——控制依赖。
if转换把程序中的分支语句转换为条件执行语句,从而将全部的分支语句删除,因此能够起到把控制对依赖的影响转换成数据依赖的作用。
而控制依赖则要求编译器在做代码变换的时候,除了要保持数据依赖以外,还要保持控制依赖,从而限制可以进行的变换来保证程序的正确性。
本文是一篇综述性质的文章,包含了对if转换的研究近况的调研,以及这些研究成果在LLVM中的实现情况。

The development of high performance architectures has made it harder to design an efficient complier.
There is one question in common that every state-of-the-art compilers face: when the order of two instructions can be exchanged without changing the meaning of the program.
To solve this question, most compilers adopt strategies based on dependency.
Data dependence can be defined by considering the access to the same memory location.
However, most programs contain branchings, which makes the dependence complicated.
Compilers would generate wrong codes if the branchings were not handled appropriately.
There are two ways to handle branchings: if-conversion and the introduction of control dependence.
If-conversion eliminates all the branchings in the program by replacing statements with guarded statements, therefore, converts control dependence to data dependence.
The introduction of control dependence prevents  compilers from generating incorrect codes by demanding the compilers to keep the order of statements with control dependence.
This article is a review of recent researches on if-conversion and their implementation on LLVM.
\end{abstract}

\newpage

\tableofcontents

\newpage

if转换最早由Rice大学的Allen等人提出\cite{allen1983Concondeptodatdep},

\section{第一节}

这是正文你懂吗,这是正文你懂吗,这是正文你懂吗,这是正文你懂吗,
这是正文你懂吗,这是正文你懂吗,这是正文你懂吗,这是正文你懂吗,
这是正文你懂吗,这是正文你懂吗,这是正文你懂吗,这是正文你懂吗,
这是正文你懂吗,这是正文你懂吗,这是正文你懂吗,这是正文你懂吗,
这是正文你懂吗,这是正文你懂吗,这是正文你懂吗,这是正文你懂吗,
这是正文你懂吗,这是正文你懂吗,这是正文你懂吗,这是正文你懂吗,

\subsection{第一小节}

这是正文你懂吗,这是正文你懂吗,这是正文你懂吗,这是正文你懂吗,
这是正文你懂吗,这是正文你懂吗,这是正文你懂吗,这是正文你懂吗,
这是正文你懂吗,这是正文你懂吗,这是正文你懂吗,这是正文你懂吗,
这是正文你懂吗,这是正文你懂吗,这是正文你懂吗,这是正文你懂吗,
这是正文你懂吗,这是正文你懂吗,这是正文你懂吗,这是正文你懂吗,
这是正文你懂吗,这是正文你懂吗,这是正文你懂吗,这是正文你懂吗,

这是正文你懂吗,这是正文你懂吗,这是正文你懂吗,这是正文你懂吗,
这是正文你懂吗,这是正文你懂吗,这是正文你懂吗,这是正文你懂吗,
这是正文你懂吗,这是正文你懂吗,这是正文你懂吗,这是正文你懂吗,
这是正文你懂吗,这是正文你懂吗,这是正文你懂吗,这是正文你懂吗,
这是正文你懂吗,这是正文你懂吗,这是正文你懂吗,这是正文你懂吗,
这是正文你懂吗,这是正文你懂吗,这是正文你懂吗,这是正文你懂吗,

这是正文你懂吗,这是正文你懂吗,这是正文你懂吗,这是正文你懂吗,
这是正文你懂吗,这是正文你懂吗,这是正文你懂吗,这是正文你懂吗,
这是正文你懂吗,这是正文你懂吗,这是正文你懂吗,这是正文你懂吗,
这是正文你懂吗,这是正文你懂吗,这是正文你懂吗,这是正文你懂吗,
这是正文你懂吗,这是正文你懂吗,这是正文你懂吗,这是正文你懂吗,
这是正文你懂吗,这是正文你懂吗,这是正文你懂吗,这是正文你懂吗,

\bibliography{thesis}{}
\bibliographystyle{plain}

\end{document}